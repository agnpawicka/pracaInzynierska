\chapter{Wprowadzenie}%Wstęp - opis problemu, motywacja
\section{Motywacja}
Przeprowadzanie testów i ankiet jest wpisane w życie akademickie oraz szkolne. Jedną ze szczególnych metod przeprowadzania testu lub ankiety jest weryfikowanie wiedzy czy jej zdalne zdobywanie.
Aby to jednak było wartościowe, należy spełnić warunki takie jak:
\begin{itemize}
\item czytelność formy
\item możliwość weryfikownia, kto wysłał odpowiedź
\item kontrolowanie czasu wysyłanych odpowiedzi
\item zapamiętywanie przesłanych odpowiedzi
\end{itemize}
Oprócz powyższych kryteriów możliwym byłoby wymienienie jeszcze wielu kwestii, których implementacja znacznie ułatwiłaby pracę ze zdalnymi testami - zwłaszcza dla strony przeprowadzającej test, jak np.:
\begin{itemize}
\item automatyczne generowanie testu z popularnego formatu,
\item możliwość wstawiania symboli matematycznych,
\item automatyczne ocenianie (dla testow)  - być może w nietypowej skali,
\item generowanie losowych testów z pytań z pewnej puli,
\item udostępnianie wyników w wygodnym formacie
\end{itemize}
Zupełnie nowe narzędzie - możliwie blisko ideału - byłoby więc ambitnym przedsięwzięciem dla zespołu programistów. Tu z pomocą przychodzi możliwość rozszerzania istniejących rozwiązań.

\section{Gotowe rozwiązania}
Do najpopularniejszych gotowych rozwiązań należą formularze Googla, oraz Microsoftu - udostępniają one już bardzo dużo wymienionych w poprzednim rozdziale możliwości - są czytelne, zapewniają weryfikowanie  tożsamości osób przesyłających odpowiedź (sprawdzają zalogowanego kontem mailowym użytkownika, w Google Forms dzieje się  to przez OAuth).
\ind Wymienione narzędzia wciąż jednak nie są idealne. Głównymi problemem w przeprowadzaniu większych testów/ankiet jest konieczność ręcznego ,,przeklikiwania się'' przez interfejs do tworzenia formularzy, a w dziedzinach ścisłych również brak wbudowanej interpretacji symboli matematycznych - chociaż własności, które byłyby pomocne jest znacznie więcej. Udostępnione możliwości oceniania automatycznego nie pozwalają na ustawienie nietypowej (jak np. wykładniczej) skali oceniania. Google Forms w czystej formie nie pozwalają też na automatyczne ustawienie ram czasowych akceptacji odpowiedzie - czasu początku i końca ,,testu''. Format odpowiedzi również nie jest najprostszy do przetwarzania.
\ind Jest jednak ważna zaleta gotowych rozwiązań - można do nich dobudowywać dalsze usprawnienia. Niniejsza praca polega na usprawnieniu gotowego rozwiązania jakim są Google Forms o parę nowych możliwości.
\section{Cel pracy}
 Poprzez wykorzystanie Google Apps Script  i rozwiązań serwerowych bibliotek JavaScriptu powstało niewielkie usprawnienie działania znanego już rozwiązania - formularzy Google. W prezentowanej pracy zostało zaiplementowane automatyczne generowanie formularzy z plików w formacie JSON, renderowanie wstawek napisanych w \LaTeX{}'u oraz możliwość prostego włączania oraz wyłączania opcji, mówiącej czy formularz przyjmuje w danym momencie zgłoszenia. Niektóre opcje formularzy można  przekazać już na podstawie kodowania JSONowego - w konstruowanym pliku.

