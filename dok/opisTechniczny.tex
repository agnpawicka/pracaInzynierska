\chapter{Techniczny opis programu}
Narzędzie składa się z trzech głównych części:
\begin{itemize}
\item interfejsu użytkownika napisanego przy użyciu frameworku css'owego ,,Bootstrap'',
\item aplikacji internetowej, napisanej we frameworku Google Apps Script po stronie Google'a, pod prywatnym kontem e-mailowym, 
\item serwera napisanego w Node.js. Serwer jest uruchamiany lokalnie na urządzeniu użytkownika, komunikuje się z aplikacją internetową oraz interfejsem użytkownika.

\end{itemize}
Poza tym lokalny serwer wykorzystuje kilka bibliotek JavaScriptowych oraz Pythonowych do dodatkowych obliczeń. Schemat połączeń w projekcie:
\\tu będzie schemat narysowany\\%TODO
\section{Interfejs użytkownika}%TODO napisać o json sxhema i zlinkować następny rozdział (instrukcja, schemat
Na interfejs użytkownika składają się trzy pliki: \textit{strona.html}, \textit{memoryFile.js} oraz \textit{htmlCode.js} - pierwszy z nich koduje część wizualną, drugi dane na temat formularzy, trzeci jest odpowiedzialny za zachowanie poszczególnych elementów, w tym za komunikację z serwerem. 
\ind Na wygląd aplikacji składają się:
\begin{itemize}
\item pole do wgrywania plików, przyjmujące formaty .json oraz .txt. Zawartość powinna spełniać wymagania opisane w instrukcji w podrozdziale 4.2, %TODO link zrobić.
\item przycisk wgrywający plik - ,,generate form'' wywołuje metodę getFile,
\item listę formularzy, w której poprzez kliknięcie można wybrać formularz,
\item kilka przycisków powiązanych z metodą getInfo().
\end{itemize}
\ind Plik \textit{memoryFile.js} jest modyfikowany na poziomie serwera lokalnego. Jako kod zawiera on wyłacznie deklarację stałej ,,memory'', będącej obiektem JSON następującej budowy:
\begin{lstlisting}[language=json,firstnumber=1]
{"forms":
    [{"id": "string",
      "date": "string",
      "name": "string"}
    ]
}
\end{lstlisting}
Wartość ,,id'' jest generowana przez Google przy tworzeniu formularza, ,,date'' to data utworzenia formularza (rok-miesiąc-dzień), ,,name'' odpowiada tytułowi.
\ind Wartość ,,selectedForm'' z pliku \textit{htmlCode.js} odpowiada identyfikatorowi formjlarza obecnie podświetlonego na niebiesko na liście formularzy. Metody zaimplementowane w \textit{htmlCode.js}:
\begin{itemize}
\item changeLocation(newLocation) - pozwala na zmianę wyświetlanego adresu,
\item makeHttpRequest(Url, callback) - wysyła rządanie GET do lokalnego serwera (adres URL zależny jest od rodzaju przeprowadzanej akcji), w przypadku pozytywnej odpowiedzi (HTTP status 200) wywołuje metodę callback,
\item getInfo(action) - metoda przypisana większości przycisków interfejsu, jest odpowiedzialna za odpowiednie wywołanie makeHttpRequest,
\item getFile() - funkcja przypisana do przycisku ,,Generate form'', pobiera wgrany plik, sprawdza, czy jest on poprawnym JSONem i wywołuje makeHttpRequest z odpowiednimi argumentami.
\item assign(id) - metoda przypisana do elementów listy formularzy, ustawia wartość ,,selectedForm'' przy każdej zmianie wybranego z listy formularza,
\item generateList() - tworzy listę formularzy na podstawie \textit{memoryFile.js}, jest uruchamiana przy ładowaniu strony.
\end{itemize}

\section{Google Apps Script}
Aplikacja internetowa stanowi rozwiązanie serwerowe, pozwalające na komunikację poprzez HTTP. Odwołać do niej może się każdy użytkownik znający adres url aplikacji, kod wykonywany jest pod kontem Google właściciela aplikacji (w chwili obecnej aplikacja jest utworzona pod prywatnym kontem Google).

\ind Po stronie Google'a znajdują się dwa pliki: \textit{communication.gs} oraz \textit{createFrom.gs}, odpowiadające kolejno za komunikację po HTTP oraz za zarządzanie tworzeniem formularzy. 
\subsection{HTTP}
Plik \textit{communication.gs} zawiera implementację metod doPost(e)oraz doGet(e). Framework zapewnia, że są one wykonywane, gdy do aplikacji przyjdą zapytania HTTP (odpowiednio POST i GET). Do parametrów funkcji są przekazywane treści zapytań.
\ind Metoda doGet(e) jest wykorzystywana do zapytań dotyczących tego, czy dany formularz przyjmuje przesyłane odpowiedzi oraz do uzyskiwania informacji na temat adresów URL danego formularza.  Poprawne zapytanie powinno zawierać dwie wartości:
\begin{itemize}
\item formId - identyfikator formularza przypisywany automatycznie w momencie tworzenia,
\item action - pole tekstowe mówiące o tym, co autor zapytania chce zrobić. Obsługiwane wartości:
\begin{itemize}e
\item \textit{isActive} zwraca wiadomość tekstową informującą o tym, czy formularz przyjmuje odpowiedzi,
\item \textit{activate} aktywuje formularz (po wykonaniu tej częsci formularz przyjmuje odpowiedzi),
\item \textit{deactivate} dezaktywuje formularz,
\item \textit{publisherUrl} zwraca adres url formularza dla respondentów,
\item \textit{editorUrl} zwraca adres url do edycji formularza,
\item \textit{delete}.
\end{itemize}
\end{itemize}
Z braku udostępnionej metody usuwającej rządany formularz - zapytanie o usunięcie formularza dezaktywuje go. 
\ind Metoda doPost(e) odpowiada  za tworzenie nowych formularzy. Przesyłana zawartość zawiera zakodowany w formacie JSON formularz, wstępnie przetworzony przez lokalny serwer (zamiana pytań z wstawkami \LaTeX{}'owymi na zdjęcia w formacie base64). Ta metoda uruchamia funkcję \textit{createFromJSON} z pliku \textit{createForm.gs}.
\subsection{Zarządzanie formularzami}
Plik \textit{createForm.gs} zawiera kilka metod służących do tworzenia konkrentych rodzajów pytań:
\begin{itemize} %TODO rodzaje pytań
\item checkBox - 
\end{itemize}
Pytania ze wstawkami \LaTeX{}'owymi są w formie zdjęć, nie posiadają możliwych odpowiedzi - tuż pod nimi tworzy się pytanie bez treści zawierające możliwe odpowiedzi. W praktyce wygląda to w następujący sposób: (TU BĘDZIE ZDJĘCIE);
%TODO wstawić fotę
Takie rozwiązanie wynika z ograniczeń  API formularzy, które nie udostępnia metod wprowadzenia zdjęć do popularnych typów pytań - jest to znany problem, zgłaszany \href{https://issuetracker.google.com/issues/36765518?pli=1}{tutaj}. Metoda \textit{image} tworzy pole zdjęciowe w formularzu. 
Funkcja \textit{setFromFeatures} modyfikuje ustawienia formularza - chodzi tu o zapamiętywanie adresów mailowych użytkowników przesyłających odpowiedzi, limit odpowiedzi na użytkownika, ustawienie współwłaściciela formularza oraz wartości, czy formularz powinien być oceniany automatycznie.
\ind Główną funkcją w pliku jet \textit{createFromJSON}, która generuje formularz poprzez wywoływanie odpowiednich metod z wymienionych wyżej. Właścicielem formularza jest właściciel aplikacji internetowej (użytkownik Google, pod którego  kontem tworzone są formularze). Takie rozwiązanie pozwala na uniknięcie potrzeby autoryzacji przy każdym dostępie do aplikacji internetowej.
\section{Serwer Node.js'owy}
Serwer 

\section{Konwersja wstawek z\LaTeX{}'a}
\subsection{Alternatywne rozwiązanie}%o tym czemu nie mathjax

\section{Skrypty instalujący i uruchamiający}









