% Opcje klasy 'iithesis' opisane sa w komentarzach w pliku klasy. Za ich pomoca
% ustawia sie przede wszystkim jezyk i rodzaj (lic/inz/mgr) pracy, oraz czy na
% drugiej stronie pracy ma byc skladany wzor oswiadczenia o autorskim wykonaniu.
\documentclass[lic,shortabstract]{iithesis}
% Własne dodatkowe pakiety:
\usepackage{graphicx}

\usepackage[utf8]{inputenc}

%% Makra dla tytułów, które pojawiają się także w streszczeniu

%%%%% DANE DO STRONY TYTUŁOWEJ
% Niezaleznie od jezyka pracy wybranego w opcjach klasy, tytul i streszczenie
% pracy nalezy podac zarowno w jezyku polskim, jak i angielskim.
% Pamietaj o madrym (zgodnym z logicznym rozbiorem zdania oraz estetyka) recznym
% zlamaniu wierszy w temacie pracy, zwlaszcza tego w jezyku pracy. Uzyj do tego
% polecenia \fmlinebreak.
\polishtitle    {tytul}
\polishabstract {streszczenie}
% w pracach wielu autorow nazwiska mozna oddzielic poleceniem \and
\author         {Agnieszka Pawicka}
% w przypadku kilku promotorow, lub koniecznosci podania ich afiliacji, linie
% w ponizszym poleceniu mozna zlamac poleceniem \fmlinebreak

\advisor        {dr Jan Otop}
\date          {Wrocław 2021} % 4 września 2020}                     % Data zlozenia pracy
% Dane do oswiadczenia o autorskim wykonaniu
\transcriptnum {300167}                     % Numer indeksu
\advisorgen    {dr. Jana Otopa} % Nazwisko promotora w dopelniaczu
%%%%%

\begin{document}

%%%%% POCZĄTEK ZASADNICZEGO TEKSTU PRACY
\chapter{Wprowadzenie}%Wstęp - opis problemu, motywacja
\section{Motywacja}
Przeprowadzanie testów i ankiet jest wpisane w życie akademickie oraz szkolne. Jedną ze szczególnych metod przeprowadzania testu lub ankiety jest weryfikowanie wiedzy czy jej zdalne zdobywanie.
Aby to jednak było wartościowe, należy spełnić warunki takie jak:
\begin{itemize}
\item czytelność formy
\item możliwość weryfikownia, kto wysłał odpowiedź
\item kontrolowanie czasu wysyłanych odpowiedzi
\item zapamiętywanie przesłanych odpowiedzi
\end{itemize}
Oprócz powyższych kryteriów możliwym byłoby wymienienie jeszcze wielu kwestii, których implementacja znacznie ułatwiłaby pracę ze zdalnymi testami - zwłaszcza dla strony przeprowadzającej test, jak np.:
\begin{itemize}
\item automatyczne generowanie testu z popularnego formatu,
\item możliwość wstawiania symboli matematycznych,
\item automatyczne ocenianie (dla testow)  - być może w nietypowej skali,
\item generowanie losowych testów z pytań z pewnej puli,
\item udostępnianie wyników w wygodnym formacie
\end{itemize}
Zupełnie nowe narzędzie - możliwie blisko ideału - byłoby więc ambitnym przedsięwzięciem dla zespołu programistów. Tu z pomocą przychodzi możliwość rozszerzania istniejących rozwiązań.

\section{Gotowe rozwiązania}
Do najpopularniejszych gotowych rozwiązań należą formularze Googla, oraz Microsoftu - udostępniają one już bardzo dużo wymienionych w poprzednim rozdziale możliwości - są czytelne, zapewniają weryfikowanie  tożsamości osób przesyłających odpowiedź (sprawdzają zalogowanego kontem mailowym użytkownika, w Google Forms dzieje się  to przez OAuth).
\ind Wymienione narzędzia wciąż jednak nie są idealne. Głównymi problemem w przeprowadzaniu większych testów/ankiet jest konieczność ręcznego ,,przeklikiwania się'' przez interfejs do tworzenia formularzy, a w dziedzinach ścisłych również brak wbudowanej interpretacji symboli matematycznych - chociaż własności, które byłyby pomocne jest znacznie więcej. Udostępnione możliwości oceniania automatycznego nie pozwalają na ustawienie nietypowej (jak np. wykładniczej) skali oceniania. Google Forms w czystej formie nie pozwalają też na automatyczne ustawienie ram czasowych akceptacji odpowiedzie - czasu początku i końca ,,testu''. Format odpowiedzi również nie jest najprostszy do przetwarzania.
\ind Jest jednak ważna zaleta gotowych rozwiązań - można do nich dobudowywać dalsze usprawnienia. Niniejsza praca polega na usprawnieniu gotowego rozwiązania jakim są Google Forms o parę nowych możliwości.
\section{Cel pracy}
 Poprzez wykorzystanie Google Apps Script  i rozwiązań serwerowych bibliotek JavaScriptu powstało niewielkie usprawnienie działania znanego już rozwiązania - formularzy Google. W prezentowanej pracy zostało zaiplementowane automatyczne generowanie formularzy z plików w formacie JSON, renderowanie wstawek napisanych w \LaTeX{}'u oraz możliwość prostego włączania oraz wyłączania opcji, mówiącej czy formularz przyjmuje w danym momencie zgłoszenia. Niektóre opcje formularzy można  przekazać już na podstawie kodowania JSONowego - w konstruowanym pliku.



\chapter{Środowisko} 
Program składa się z trzech głównych elementów:
\begin{itemize}
\item kodu rozszerzającego narzędzie Google Forms (opisanego w 2.1),
\item serwera zaimplementowanego w środowisku Node.js (rozdział 2.2),
\item interfejsu, napisanego przy pomocy HTML i CSS (2.3).
\end{itemize}
\ind Fragmenty kodu korzystają również z bibliotek Pythonowych - skrypt do konwersji symboli matematycznych i zdjęć.
\ind Komunikacja pomiędzy elementami pracy wygląda następująco:
\begin{figure}[H]
  \includegraphics{schemat.png}
  \caption{Schemat połączeń}
  \label{fig:1}
\end{figure}

\section{Google Apps Script}
Nabudowywanie na gotowym narzędziu wymaga dostępu do niego. Google udostępnia API operujące na całym szeregu klas i metod oferowanych usług. Kod przypisany jest do danego konta Google, możliwe jest ustawienie parametrów,  takich  jak: kto ma dostęp do nowotworzonej aplikacji internetowej (właściciel, zalogowany użytkownik danej organizacji, dowolny zalogowany użytkownik, każdy) pod czyim kontem jest ona uruchamiana (właściciela/współwłaściciela, czy też zalogowanego użytkownika). 

\paragraph{Framework}
Udostępnione API w pewnym stopniu wymusza na użytkownikach, aby kod wykonywany na infrastrukturze Google'a był napisany w JavaScripcie, we framework'u ,,Google Apps Script''. Program jest wykonywany po stronie serwera. Pozwala on na stosunkowo łatwe manipulowanie działaniem produktów Google takich jak formularze, arkusze, dysk i inne. 
\ind Framework powstał w 2009 roku, w JavaScript 1.6, jest jednak regularnie ulepszany.

\paragraph{Komunikacja}
 Apps Script udostępnia komunikację przez protokół HTTP - jeśli projekt zawiera funkcje doGet(e) / doPost(e) - odpowiednie żądania wykonują kod z ciała tych metod. Zwracane wartości prowadzą do przekierowań zapytań  - automatycznie tworzony jest nowy adres URL. Wysłanie żądania GET pozwala na dotarcie do potrzebnych danych. 
\paragraph{API}
Google Apps Script udostępnia szereg klas i metod związanych z poszczególnymi narzędziami. Szczegółowa dokumentacja narzędzia znajduje się tutaj: \href{https://developers.google.com/apps-script/reference/forms}{Forms Service}.
 Główna klasa - FormApp - jest odpowiedzialna za zarządzanie formularzami - m.in. tworzenie nowych. Każdy typ pytania i element formularza ma odpowiednią klasę (jak np. CheckboxItem czy SectionHeaderItem). Poprzez klasę From  można zmieniać głowne ustawienia formularzy  - jak na przykład dodawanie właścicieli, tworzenie pytań, automatyczne ocenianie, manipulowanie tytułem, ustawienie, czy formularz jest ,,aktywny'' (czy przyjmuje odpowiedzi).
 
 
\section{Node.js}
Node.js jest środkowiskiem uruchomieniowym JavaScriptu - służącym do tworzenia aplikacji serwerowych. 
Praca wykorzystuje kilka bibliotek, przede wszystkim korzysta jednak z możliwości serwerowych Node.js. 
\paragraph{http}
Interfejs Node.js silnie związany z ,,sercem'' środowiska - udostępnia narzędzia do komunikacji poprzez protokół  HTTP zarówno ze strony serwerowej jak i klienckiej. W pracy w czystej formie wykorzystywany do wysyłania zapytań pomiędzy serwerem lokalnym a serwerem Google'a.
\newline Dokumentacja: \href{https://nodejs.org/api/http.html}{http}
\paragraph{express}
,,Szybki (...), minimalistyczny framework webowy dla Node.js'' - narzędzie pozwala w przystępny sposób postawić serwer (korzysta z biblioteki http). Udostępnia cztery klasy:
\begin{itemize}
\item application - odpowiada aplikacji serwerowej,
\item  request - zarządza odwołaniami do serwera (głównie parametrami),
\item response - odpowiada za odpowiedzi serwera,
\item router - zajmuje się routingiem, może być używane jako oprogramowanie pośredniczące.
\end{itemize}
Trzon kodu narzędzia opiera się właśnie na serwerze express'owym.
\ind Dokumentacja znajduje się tutaj: \href{https://expressjs.com/}{expres.js}
\paragraph{cors}
Node.js'owy moduł pozwalający na odpowiednie ustawienia CORS (Cross-Origin Request Sharing) w rozwiązaniach typu express. CORS jest metodą rozwiązania problemów z domyślnymi ustawieniami związanymi z bezpieczeństwem. Standardowo dane z jednej strony mogą być pobierane z poziomu drugiej strony gdy obie strony są z tego samego źródła (ten sam schemat Url, host oraz port). CORS pozwala na uniknięcie tej konieczności.
\ind Github modułu: \href{https://github.com/expressjs/cors}{cors}
\paragraph{child\_process}
Moduł Node.js'owy pozwalający na uruchamianie podprocesów. W przypadku omawianego kodu, umożliwa uruchamianie skryptów napisanych  w Pythonie z poziomu kodu Node.js'owego.
\ind Dokumentacja znajduje się tutaj: \href{https://nodejs.org/api/child_process.html}{child\_process}
\paragraph{jsonschema}
\ind Nowy (zaledwie dziesięciomięczny, wciąż w wersji Beta) moduł pozwalający na walidację formatu json zgodnie z podanym schematem.
\ind Oficjalna strona: \href{https://www.npmjs.com/package/jsonschema}{jsonschema}

\section{Python}
Rozwiązania pythonowe zostały wykorzystane w celu konwersji wstawek matematycznych (\LaTeX{}) do zdjęć. Poniżej krótki opis wykorzystanych bibliotek.
\paragraph{tex2pix} Biblioteka pozwalająca na konwertowanie formatu .tex do różnych formatów zdjęciowych. Metoda konwertująca format .tex na format .png zaczyna od konwersji .tex do .pdf, stąd w pracy używana jest konwersja do pdf z tej biblioteki, a dalsze manipulowanie formatem używa innych - subiektywnie prostszych w użytkowaniu - bibliotek. 
\ind Oficjalna strona: \href{https://pypi.org/project/tex2pix/}{tex2pix}.

\paragraph{pdf2image} Biblioteka  umożliwiająca konwersję formatu pdf do formatów zdjęciowych.
\ind Oficjalna strona: \href{https://pypi.org/project/pdf2image/}{pdf2image}.
\paragraph{opencv} Biblioteka pozwalająca na manipulację obrazami. Udostępnia znacznie więcej możliwości niż te użyte w pracy. W szczególności pozwala na przycinanie obrazów względem ich kolorystyki - co pozwala na automatyczne przycięcie strony pdf do rozmiarów napisanego na niej tekstu. Więcej informacji na temat biblioteki: \href{https://pypi.org/project/opencv-python/}{opencv}
\paragraph{base64} Biblioteka pozwala na konwersję obrazu do formatu Base64 - używanego w pracy do przesyłu obrazów pomiędzy serwerami node'owym a google'owym. 

\section{Bootstrap}  
Popularna biblioteka CSS, ułatwiająca budowanie interfejsów graficznych stron internetowych pisanych w HTML. Oficjalna strona \href{https://getbootstrap.com/}{bootstrap}.





\chapter{Techniczny opis programu}
\section{Google Apps Script}
\subsection{HTTP}
\subsection{Zarządzanie formularzami}
\section{Interfejs użytkownika}
\section{Serwer Node.js'owy}

\section{Konwersja wstawek z\LaTeX'a}
\subsection{Alternatywne rozwiązanie}%o tym czemu nie mathjax
 
\section{Skrypty instalujący i uruchamiający}


\chapter{Instrukcja użytkownika}
\section{Początki pracy  z narzędziem}
Narzędzie pozwala na wygenerowanie formularza Google Forms zakodowanego w formacie JSON, automatyczne generowanie zdjęć na podstawie wstawek w latex'u oraz zarządzanie informacjami o tym, czy dany formularz przyjmuje przesyłane odpowiedzi. Na początek należy jednak pobrać i zainstalować zależności wykorzystywane w pracy.
\subsection{Instalacja}
Na początku należy sklonować \href{https://github.com/agnpawicka/pracaInzynierska/}{repozytorium projektu}.\\
Następnie wejść w folder \textbf{source} oraz uruchomić plik o nazwie ,,install.sh''. Pozwoli to na zainstalowanie potrzebnych bibliotek.
\subsection{Uruchomienie}
Aby uruchomić narzędzie należy uruhomić plik o nazwie ,,run.sh''. Uruchomi on lokalny serwer Node.js'owy oraz stronę internetową z interfejsem użytkownika.
\section{Schemat pliku kodującego (JSON)}

\section{Obsługa narzędzia}
Po uruchomieniu użytkownik widzi stronę w przeglądarce, jak na zdjęciu poniżej:
\begin{figure}
  \includegraphics{strona.png}
  \caption{Interfejs aplikacji}
  \label{fig:1}
\end{figure}
\\Widoczne u góry pole do wgrywania plików przyjmuje formaty .txt oraz .json. W pliku powinien znajdować się zakodowany formularz (podrozdział 4.2. Schemat pliku kodującego (JSON)).
\\Poniżej  po lewej stronie znajduje się lista utworzonych już formularzy, po prawej znajdują się przyciski służące do operowania na już istniejących formularzach.
\\Przycisk ,,Generate form'' wgrywa podany plik wykonuje na nim kolejne operacje. Poprawne wykonanie powinno przechodzić przez kolejne etapy:
\begin{itemize}
\item Sprawdzany jest format -  pliku. Jeśli zawartość jest obiektem typu JSON, dane przekazywane są do lokalnego serwera, w przeciwnym przypadku strona wyświetli alert informujący o niepoprawnym formacie.
\item Serwer lokalny sprawdza zgodność pliku ze schematem (JSON schema). Po tym etapie poniżej pola do wgrywania plików powinna pojawić się jedna z poniższych informacji:
\begin{itemize}
\item Validation succeded
\item Wrong JSON format
\end{itemize}
\item Jeśli JSON jest zgodny ze schematem, następuje konwertowanie pytań zakodowanych jako ,,tex'' na format zdjęciowy. 
\item Po zakończonej konwersji z lokalnego serwera wysyłany jest POST request do serwera po stronie Google, gdzie trwa konwersja pliku na formularz. Zdalny serwer odsyła informację po zakończonej pracy do serwera lokalnego.
\item Lokalny serwer dodaje nowy formularz do listy formularzy.
\item Wyświetla się komunikat \textbf{New form has beed created. Please reload page} z prośbą o odświeżenie strony.
\end{itemize}
\\Zachowania poszczególnych przycisków  - za wyjątkiem ,,Generate Form'' - dotyczą zawsze wybranego formularza z listy (podświetlonego w danym momencie na niebiesko). Aby wybrać formularz należy kliknąć na niego w liście formularzy. 
\paragraph{Is Form Ative} przycisk zwraca wartość \textbf{Form is active} jeśli formularz przyjmuje odpowiedzi oraz \textbf{Form is inactive} w przeciwnym przypadku.
\paragraph{Activate} Wysyła do serwera po stornie Google'a zapytanie, jeśli aktywacja formularza przebiegła pomyślnie, zwracana jest wiadomość \textbf{Form activated}.
\paragraph{Deactivate} Wysyła do serwera po stornie Google'a zapytanie, jeśli dezaktywacja formularza przebiegła pomyślnie, zwracana jest wiadomość \textbf{Form deactivated}.
\paragraph{Delete Form} Wysyła do serwera po stronie Google'a zapytanie o dezaktywnację formularza, następnie usuwa z pliku z danymi o formularzach wpis dotczący wybranego formularza oraz w komunikacie \textbf{Form deactivated, please reload page.} prosi o odświeżenie strony
\paragraph{Get Publisher Url} Wysyła do serwera po stronie Google'a zapytanie o adres url dla respondentów wybranego formularza. W komunikacie pojawia się odpowiedni link.
\paragraph{Get Editor Url} Wysyła do serwera po stronie Google'a zapytanie o adres url dla edytorów wybranego formularza. W komunikacie pojawia się odpowiedni link.



\chapter{Podsumowanie}%problemy i rozwój o konkluzje

\section{Wnioski}
Celem pracy była implementacja narzędzia pozwalającego na uproszczenie przeprowadzania testów i ankiet online w oparciu o istniejące już rozwiązanie - Google Forms.  Efektem jest program, ułatwiający pracę z formularzami Google'a poprzez umożliwienie automatycznego gemerowania pytań oraz konwersję symboli matematycznych.  Możliwe jest jednak dalsze rozwijanie narzędzia. 

\section{Rozwój}
Możliwych rozszerzeń jest wiele - niniejszy rozdział opisuje kilka z nich.
\paragraph{Zarządzanie przesłanymi odpowiedziami} - Google Apps Script udostępnia klasę ItemResponse - odpowiedzialną za zarządzanie odpowiedziami respondentów. Możliwe jest przechwytywanie odpowiedzi, oceny (wystawionej automatycznie, jeśli tak ustawiono), pytania, do którego jest odpowiedz (jako klasy, a wiec razem z możliwymi odpowiedziami, punktami, treścią) a także ustawienie komentarza. Dalej: możliwe jest przetwarzanie pozyskanych danych i ocenianie testów przez zewnętrzny program na niestandardowych zasadach (jak na przykład wykładnicza skala punktowa w zależności od liczby poprawnych odpowiedzi).
\paragraph{Możliwość ustawiania losowej kolejności pytań} - choć w teorii ten problem jest rozwiązany przez udostępnione API, w praktyce pojawia się problem. Jak wspomniano wyżej - API formularzy nie udostępnia metod wprowadzenia zdjęć do popularnych typów pytań. Powoduje to konieczność utrzymywania dwóch formalnie osobnych pytań (w praktyce pytania-zdjęcia i odpowiedzi) w jednym miejscu w formularzu.
\paragraph{Ustawianie czasu rozpoczęcia oraz zakończenia testu} - aby mogło się to odbywać automatycznie, potrzebne jest urządzenie, które będzie reguarnie i w krótkich odstępach czasowych sprawdzało, czy należy już wysłać odpowiednie zapytanie do zaimplementowanej aplikacji, czy jeszcze nie. Może się to wykonywać po stronie serwera lokalnego lub aplikacjii po stronie Google'a. Zaimplementowany serwer nie uzwględnia jednak tej możliwości ze względu na konieczność ciągłego trwania w stanie uruchomionym. 



%%%%% BIBLIOGRAFIA


\appendix

\end{document}
