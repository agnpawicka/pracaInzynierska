\documentclass{article}
\usepackage[utf8]{inputenc}
\usepackage[OT4]{fontenc}
\usepackage[polish]{babel}

\begin{document}

\section{zadanie 1}
\fbox{\begin{minipage}{35em}
Jasio podrzucił talerz do góry nadając mu prędkość początkową 10m/s. Po jakim czasie talerz zmienił kierunek (zaczął spadać?)
\end{minipage}}
\section*{zadanie 2}
Weźmy talerz z poprzedniego zadania. Jak wysoko doleciał talerz?  
\section*{zadanie 3}
Pocisk wystrzelono z karabinu AK-47 (czyli jak można się łatwo dowiedzieć z internetu jego prędkość początkowa wynosiła 715 m/s). Cel stoi 1,5 km od miejsca wystrzału. Były opory powietrza, wietrzyk wiał w twarz - przyjmijmy, że pocisk miał spowolnienie 100 $\frac{m}{s^{2}}$. Z jaką prędokością walnie w cel?
\section*{zadanie 4}
Pewien kierowca wjeżdża na autostradę z prędkością 80 km/h. Już po 30 sekundach ma prędkość 120 km/h. Jakie miał przyspieszenie?

\section*{zadanie 5}
Maciuś puszcza autka po podłodze. Jednemu nadał prędkość (czyli począkową) 2 m/s. Korytarz jest długi więc auto nie walnęło w żadną ścianę tylko zatrzymało się spokojnie 4 sekundy później. Jaką prędkość miało po 3 sekundach ruchu?
\section*{zadanie 6}
Mała Lenka goni Ćwiartkę wokół stołu. Na początku jest od niej 1m. Już po sekundzie Ćwiartka jest po przeciwnej stronie stołu niż Lenka - czyli powiedzmy 5 metrów od Lenki. O ile szybciej biegnie Ćwiartka od Lenki? (obie biegną jednostajnie)
\section*{zadanie 7}
Rowerzysta jechał na rowerze ze stałą prędkością 25 km/h. Niestety nie zauważył ściany i zaczął hamować 6 m przed nią. Jego przyspieszenia w czasie hamowania wynosiło -4 $\frac{m}{s^{2}}$. Gdzie się zatrzymał? (Ile od ściany).
\section*{zadanie 8}
Natalka właśnie nauczyła się jeździć na rowerze i postanowiła pościgać się z biegnącą mamą. Obie zaczęły z prędkością 0 m/s. Po sekundzie Natalka miała prędkość 1m/s a mama 0,5 m/s. Po drugiej sekundzie Natalka miała wciąż prędkość 1 m/s, a mama 1 m/s. Potem już żadna nie przyspieszała. Tor wyścigowy miał 30 metrów. Ile czasu zajęło osiągnięcie mety mamie, a ile Natalce? 
\section*{zadanie 9}
Jedzie sobie formuła jeden (albo Zygzak McQueen) po prostej drodze. Jako że się ściga, to od startu przyspiesza. Na początku jedzie przez 10 s z przyspieszeniem 9 $\frac{m}{s^{2}}$, potem nie przyspiesza przez 15 sekund i jako że minął metę zaczyna hamować, z przyspieszeniem 12 $\frac{m}{s^{2}}$ aż się zatrzyma. Jaką odległość przejechał od startu?
\\Podpowiedź: rozbij sobie to zadanie na 3 części. Policz drogę tylko jak przyspieszał, drugą drogę jak jechał jednostajnie i trzecią jak zwalniał i dodaj je do siebie ;)
\section*{zadanie 10}(z gwiazdką, ale spróbuj ;) a nuż się uda. Jakby było ,,tragiczne'', to zatrzymaj muchę na początek - uznaj, że leci z prędkością 0 m/s)\\
Była sobie mucha, i leciała nad torami ze stałą prędkością 1 m/s. Nie wiedziała, że za nią ze stacji wyjechał pociąg (najpierw stał na stacji, potem jednostajnie przyspieszonym z przyspieszeniem 10 $\frac{m}{s^{2}}$ jechał przez 2,5 sekundy, następnie jechał ruchem jednostajnym). Po jakim czasie pociąg dogoni muchę, jeśli zaczynał 100 metrów za muchą?



\end{document}
