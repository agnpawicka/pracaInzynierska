% Opcje klasy 'iithesis' opisane sa w komentarzach w pliku klasy. Za ich pomoca
% ustawia sie przede wszystkim jezyk i rodzaj (lic/inz/mgr) pracy, oraz czy na
% drugiej stronie pracy ma byc skladany wzor oswiadczenia o autorskim wykonaniu.
\documentclass[lic,shortabstract]{iithesis}

\usepackage[utf8]{inputenc}

%% Makra dla tytułów, które pojawiają się także w streszczeniu
\newcommand{\BSaLC}{,,Build systems {\`a} la carte''}
\newcommand{\BSaLCTP}{,,Build systems {\`a} la carte: Theory and practice''}

%%%%% DANE DO STRONY TYTUŁOWEJ
% Niezaleznie od jezyka pracy wybranego w opcjach klasy, tytul i streszczenie
% pracy nalezy podac zarowno w jezyku polskim, jak i angielskim.
% Pamietaj o madrym (zgodnym z logicznym rozbiorem zdania oraz estetyka) recznym
% zlamaniu wierszy w temacie pracy, zwlaszcza tego w jezyku pracy. Uzyj do tego
% polecenia \fmlinebreak.
\polishtitle    {tytul}
\polishabstract {streszczenie}
% w pracach wielu autorow nazwiska mozna oddzielic poleceniem \and
\author         {Agnieszka Pawicka}
% w przypadku kilku promotorow, lub koniecznosci podania ich afiliacji, linie
% w ponizszym poleceniu mozna zlamac poleceniem \fmlinebreak
\advisor        {dr Jan Otop}
\date          {Wrocław 2021} % 4 września 2020}                     % Data zlozenia pracy
% Dane do oswiadczenia o autorskim wykonaniu
\transcriptnum {300167}                     % Numer indeksu
\advisorgen    {dr. Jana Otopa} % Nazwisko promotora w dopelniaczu
%%%%%


\begin{document}

%%%%% POCZĄTEK ZASADNICZEGO TEKSTU PRACY

\section{Wprowadzenie}%Wstęp - opis problemu, motywacja
Przeprowadzanie testów i ankiet jest wpisane w życie akademickie i szkolne. Jedną ze szczególnych metod przeprowadzania testu/ankiety jest sprawdzanie/zdobywanie wiedzy zdalne.
Aby to jednak było wartościowe, należy zachować pewne warunki. Do takich warunków należą m. in.:
\begin{itemize}
\item czytelność formy
\item możliwość weryfikownia, kto wysłał odpowiedź
\item kontrolowanie czasu wysyłanych odpowiedzi
\item zapamiętywanie odpowiedzi
\end{itemize}
Oprócz powyższych kryteriów możliwym byłoby wymienienie jeszcze wielu kwestii, których implementacja znacznie ułatwiłaby pracę ze zdalnymi testami - zwłaszcza ze strony test przeprowadzającej, jak np.:
\begin{itemize}
\item automatyczne generowanie testu z popularnego formatu,
\item możliwość wstawiania symboli matematycznych
\item automatyczne ocenianie (dla testow)  - być może w nietypowej skali
\item generowanie losowych testów z pytań z pewnej puli
\item udostępnianie wyników w wygodnym formacie
\end{itemize}
Zupełnie nowe narzędzie - możliwie blisko ideału - byłoby więc ambitnym przedsięwzięciem dla zespołu programistów. Tu z pomocą przychodzi możliwość modyfikowania istniejących rozwiązań.

\subsection{o gotowcach}
Do najpopularniejszych gotowych rozwiązań należą formularze googla, oraz microsoftu - udostępniają one już bardzo dużo wymienionych w poprzednim rozdziale możliwości - są czytelne, weryfikowanie osób przesyłających odpowiedź mają zapewnione (w Google Forms dzieje się  to przez OAuth)


 Wymienione narzęcia wciąż jednak nie są idealne. Głównymi problemem w przeprowadzaniu większych testów/ankiet jest konieczność ręcznego ,,przeklikiwania się'' przez interfejs do tworzenia formularzy, a w dziedzinach ścisłych również brak wbudowanej interpretacji symboli matematycznych - chociaż własności, które byłyby pomocne jest znacznie więcej. 
\subsection{dlaczego google, co to jest to co piszę}

\section{Środowisko: API googla, Javascript}
\subsection{API google}
\subsection{podejścia do problemu   }
\section{Opis programu (techniczny)}
\section{Instrukcja użytkownika}
\section{Konklucje - problemy, rozwój}

%\input{6-summary}

%%%%% BIBLIOGRAFIA


\appendix
\input{a-source_notes}

\end{document}
